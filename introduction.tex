\chapter{Introduction}
\flushleft

\begin{quotation}
Yes, my friends, I believe that water will one day be employed as fuel,
that hydrogen and oxygen which constitute it, used singly or together, will
furnish an inexhaustible source of heat and light, of an intensity of which
coal is not capable. Some day the coalrooms of steamers and the tenders of
locomotives will, instead of coal, be stored with these two condensed
gases, which will burn in the furnaces with enormous calorific power. There
is, therefore, nothing to fear. 
\raggedleft Jules Verne, \emph{The Mysterious Island}\cite{vernemysterious}
\end{quotation}

The Jules Verne quotation from \emph{The Mysterious Island}  probably  represents the earliest mention of using hydrogen as fuel. This idea is still very much viable and attracts plenty of interest from both academia and industry alike. This research is motivated by the knowledge that fossil fuels are a limited resource and climate change as a result of the accumulation of greenhouse gases in the atmosphere. Recently, the CO$_2$ content in the atmosphere reaching average levels above 400 ppm for the first time in measurable history*. The accumulation has lead to intensive research into the long term storage of green-house gases, most notably CO$_2$*.

What seems to be the main issue in both problems is the storage of large quantities of gas at low pressures and low weight. This issue is of particular interest when considering using gas fuels for automotive applications as large volume or heavy containers will seriously hamper performance. Current systems employ high pressures or cooling to achieve the desired energy content, putting high demands on the storage tanks and engine and representing additional risks to traditional fuelled engines in case of catastrophic failure.

A solution to these problems would be to use high surface area materials with low densities. These materials have the capability of storing large amounts of gas due to the physisorption of gas molecules to surfaces at lower pressures and higher temperatures than currently available. With this in mind, the research into highly porous structures has been intensified in the later years. Examples of these materials include activated carbon, another example and another example, materials of which a gram of the substance has surface areas exceeding that of entire football fields.

One group of materials that in the latter years has grabbed the attention are metal organic frameworks (MOF) %need a glossary!!!
A metal-organic framework is usually made up of a metal ion that coordinates one or several bidentate, organic ligands. The resulting structures are often porous and show remarkable gas adsorption properties.\supercite{Cheetham99, Lewis09} The structure sometimes also free metal sites that allow for preferential binding of certain species.